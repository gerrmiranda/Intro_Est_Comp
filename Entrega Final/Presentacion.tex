% Options for packages loaded elsewhere
\PassOptionsToPackage{unicode}{hyperref}
\PassOptionsToPackage{hyphens}{url}
%
\documentclass[
  9pt,
  ignorenonframetext,
]{beamer}
\usepackage{pgfpages}
\setbeamertemplate{caption}[numbered]
\setbeamertemplate{caption label separator}{: }
\setbeamercolor{caption name}{fg=normal text.fg}
\beamertemplatenavigationsymbolsempty
% Prevent slide breaks in the middle of a paragraph
\widowpenalties 1 10000
\raggedbottom
\setbeamertemplate{part page}{
  \centering
  \begin{beamercolorbox}[sep=16pt,center]{part title}
    \usebeamerfont{part title}\insertpart\par
  \end{beamercolorbox}
}
\setbeamertemplate{section page}{
  \centering
  \begin{beamercolorbox}[sep=12pt,center]{part title}
    \usebeamerfont{section title}\insertsection\par
  \end{beamercolorbox}
}
\setbeamertemplate{subsection page}{
  \centering
  \begin{beamercolorbox}[sep=8pt,center]{part title}
    \usebeamerfont{subsection title}\insertsubsection\par
  \end{beamercolorbox}
}
\AtBeginPart{
  \frame{\partpage}
}
\AtBeginSection{
  \ifbibliography
  \else
    \frame{\sectionpage}
  \fi
}
\AtBeginSubsection{
  \frame{\subsectionpage}
}
\usepackage{amsmath,amssymb}
\usepackage{iftex}
\ifPDFTeX
  \usepackage[T1]{fontenc}
  \usepackage[utf8]{inputenc}
  \usepackage{textcomp} % provide euro and other symbols
\else % if luatex or xetex
  \usepackage{unicode-math} % this also loads fontspec
  \defaultfontfeatures{Scale=MatchLowercase}
  \defaultfontfeatures[\rmfamily]{Ligatures=TeX,Scale=1}
\fi
\usepackage{lmodern}
\usecolortheme{beaver}
\usefonttheme{professionalfonts}
\ifPDFTeX\else
  % xetex/luatex font selection
\fi
% Use upquote if available, for straight quotes in verbatim environments
\IfFileExists{upquote.sty}{\usepackage{upquote}}{}
\IfFileExists{microtype.sty}{% use microtype if available
  \usepackage[]{microtype}
  \UseMicrotypeSet[protrusion]{basicmath} % disable protrusion for tt fonts
}{}
\makeatletter
\@ifundefined{KOMAClassName}{% if non-KOMA class
  \IfFileExists{parskip.sty}{%
    \usepackage{parskip}
  }{% else
    \setlength{\parindent}{0pt}
    \setlength{\parskip}{6pt plus 2pt minus 1pt}}
}{% if KOMA class
  \KOMAoptions{parskip=half}}
\makeatother
\usepackage{xcolor}
\newif\ifbibliography
\setlength{\emergencystretch}{3em} % prevent overfull lines
\providecommand{\tightlist}{%
  \setlength{\itemsep}{0pt}\setlength{\parskip}{0pt}}
\setcounter{secnumdepth}{-\maxdimen} % remove section numbering
\usepackage{caption} \captionsetup[figure]{font=scriptsize} \renewcommand{\figurename}{Figura}
\ifLuaTeX
  \usepackage{selnolig}  % disable illegal ligatures
\fi
\IfFileExists{bookmark.sty}{\usepackage{bookmark}}{\usepackage{hyperref}}
\IfFileExists{xurl.sty}{\usepackage{xurl}}{} % add URL line breaks if available
\urlstyle{same}
\hypersetup{
  pdftitle={Presentacion},
  pdfauthor={Alumno},
  hidelinks,
  pdfcreator={LaTeX via pandoc}}

\title{Presentacion}
\subtitle{Ciencia de datos con R}
\author{Alumno}
\date{8 de Julio de 2020}

\begin{document}
\frame{\titlepage}

\begin{frame}{Datos}
\protect\hypertarget{datos}{}
Para la realización del trabajo se utilizó la Encuesta de Nutrición,
Desarrollo y Salud Infantil (ENDIS) 2018, del INE.

\pause

El mismo se enfocó en las siguientes medidas antropométricas de los
niños:

\begin{itemize}

  \item{Peso}
  
  \item{Talla/Altura}
  
  \item{Perímetro Cefálico}
  
  \item{Indice de Masa Corporal}
  
\end{itemize}

\pause

La idea principal es analizar el estado nutricional de los niños
uruguayos con base en dichas medidas.

\pause

Para ello fueron utilizadas las recomendaciones de la OMS (WHO 2006)
\end{frame}

\begin{frame}{Estado de Salud por Peso}
\protect\hypertarget{estado-de-salud-por-peso}{}
\end{frame}

\begin{frame}{Estado de salud por IMC}
\protect\hypertarget{estado-de-salud-por-imc}{}
\end{frame}

\begin{frame}{Estado del Perímetro cefálico}
\protect\hypertarget{estado-del-peruxedmetro-cefuxe1lico}{}
\end{frame}

\begin{frame}{Depresión en la madre durante el embarazo}
\protect\hypertarget{depresiuxf3n-en-la-madre-durante-el-embarazo}{}
\end{frame}

\begin{frame}{Lactancia a temprana edad}
\protect\hypertarget{lactancia-a-temprana-edad}{}
\end{frame}

\begin{frame}{ShinyApp}
\protect\hypertarget{shinyapp}{}
En la aplicación se presentan dos pestañas principales, una con
resultados de la ENDIS y otra con una calculadora que sigue los
estándares de crecimiento de la OMS.

\textbf{Resultados de la ENDIS} :

\begin{itemize}
    \pause
  
    \item{Gráficos de dispersión}
    
    \pause
    
    \item{Diagrama de Barras apiladas}
    
    \pause
    
    \item{Gráfico de Marimekko}
    
  \end{itemize}

\pause

\textbf{Calculadora Pediátrica} :

\pause

Si se cuenta con una hoja de cálculo con el Peso, Talla/Altura y
Périmetro Cefálico de un conjunto de niños a lo largo del tiempo, se
puede visualizar la trayectoria de crecimiento y obtener información
sobre su desarrollo.

Enlace : \url{google.com}
\end{frame}

\end{document}
